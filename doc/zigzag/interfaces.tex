\newcommand{\mymacro}[1]{{\bfseries #1}}
\newcommand{\myfunc}[1]{{\bfseries \slshape #1}}
\newcommand{\myarg}[1]{{\bfseries #1}}
\newcommand{\zigzag}{Z\raisebox{-.5ex}{i}gZ\nolinebreak\hspace{-.1667em}\raisebox{-.5ex}{a}\nolinebreak\hspace{-.125em}g }
\lstset{ language=C, numbers=left, frame=leftline }
\setlength{\unitlength}{1mm}

\section{Общее описание}

Система \zigzag представляет собой коммуникационную среду для обмена сообщениями между прикладными объектами
узлов сенсорной сети. На одном узле сети может находится несколько прикладных объектов. Прикладные объекты предназначены
, в частности, для управления датчиками ( актуаторами ) узла. Система \zigzag предоставляет объектам узла интерфейс
прикладного программирования ( API ), который описывается в данном документе. \zigzag включает в себя операционную систему 
и стек протоколов ZigBee, а также диспетчер узла. На рисунке \ref{ZigZagArch} представлена общая архитектура системы.

\begin{figure}[!h]
\centering \begin{picture}(120,94)

\put(0,0){\framebox(120,68){\bfseries \shortstack{Система\\ZigZag}}}
\put(2,2){\framebox(116,10){\large Операционная система}}

\put(2,14){\framebox(40,40){\shortstack{Стек протоколов\\ZigBee}}}
\put(78,14){\framebox(40,40){\shortstack{Диспетчер\\узла}}}
\put(2,56){\dashbox(116,10){ZigZag API}}

\put(0,70){\framebox(32,22){\shortstack{Прикладной\\объект\\$\alpha$}}}
\put(34,70){\framebox(32,22){\shortstack{Прикладной\\объект\\$\beta$}}}
\put(68,70){\framebox(18,22){\ldots}}
\put(88,70){\framebox(32,22){\shortstack{Прикладной\\объект\\$\chi$}}}

\end{picture}

\caption{Общая архитектура системы ZigZag.} \label{ZigZagArch}
\end{figure}

Стек протоколов ZigBee позволяет связать узлы в единую сеть. В свою очередь, диспетчер узла предназначен
для управления стеком ZigBee. 

Поскольку, на одном узле может быть несколько прикладных объектов, то в системе \zigzag используется
концепция портов. Каждый объект привязан к определённому порту. Все входящие сообщения перенаправляются
прикладным объектам в соответствии с их портами. Для регистрации прикладного объекта предназначен макрос
\myfunc{OBJ}, сигнатура которого представлена в листинге \ref{OBJMacro}.

\begin{lstlisting}[caption=Сигнатура макроса OBJ, label=OBJMacro]
    OBJ( 
        obj_name,
        obj_number
    )
\end{lstlisting}

{\bfseries Описание аргументов макроса \mymacro{OBJ}:}

{\itshape
\begin{enumerate}
\item \myarg{obj\_name} - имя регистрируемого прикладного объекта. 
\item \myarg{obj\_number} - уникальный номер этого прикладного объекта.
\end{enumerate}
}

Для привязки объекта класса к определённому порту предназначен макрос \myfunc{BIND}. Сигнатура 
макроса представлена в листинге \ref{BindMacro}.

\begin{lstlisting}[caption=Сигнатура макроса BIND, label=BindMacro]
    BIND( 
        obj_name,
        port_number
    )
\end{lstlisting}

{\bfseries Описание аргументов макроса \mymacro{BIND}:}

{\itshape
\begin{enumerate}
\item \myarg{obj\_name} - имя зарегистрированного прикладного объекта. 
\item \myarg{port\_number} - номер порта, к которому привязывается объект с именем \myarg{obj\_name}.
\end{enumerate}
}

Из вне прикладной объект можно представить в виде набора атрибутов и связанных с ним типов сообщений, которые
он может обрабатывать. Система \zigzag позволяет объектам запрашивать и устанавливать атрибуты друг друга.

\section{Интерфейс доступа к атрибутам}

Все атрибуты прикладного объекта должны быть зарегистрированы. Регистрация осуществляется посредством макроса 
\mymacro{ATTR} из заголовочного файла zigzag.h. В листинге \ref{AttrMacro} представлена сигнатура этого макроса.

\begin{lstlisting}[caption=Сигнатура макроса ATTR, label=AttrMacro]
    ATTR( 
        obj_name,
        attr_name, attr_number,
        attr_size, access_mode
    )
\end{lstlisting}

{\bfseries Описание аргументов макроса \mymacro{ATTR}}:

{\itshape
\begin{enumerate}
\item \myarg{obj\_name}. Имя класса, к которому принадлежит атрибут.
\item \myarg{attr\_name}. Константа, которая будет определена со значением \myarg{attr\_number}.
\item \myarg{attr\_number}. Все атрибуты прикладного объекта имеют номера из диапазона от 0 до 255.
Номера с 0 по 31 зарезервированы системой под общие атрибуты. Данные параметр задаёт номер регистрируемого атрибута.
Номера атрибутов одного объекта должны быть уникальными.
\item \myarg{attr\_size}. Размер атрибута в байтах.
\item \myarg{access\_mode}. Режим доступа к атрибуту. Допустимы следующие значения аргумента:
    \begin{itemize}
        \item A\_READ - атрибут доступен только на чтение,
        \item A\_WRITE - атрибут доступен только на запись,
        \item A\_RW - разрешено чтение и запись атрибута.
    \end{itemize}
\end{enumerate}
}

\subsection{Запрос атрибута}

Запрос значения атрибута прикладного объекта выполняется с помощью функции \myfunc{attr\_get()}.
Прикладной объект обязан предоставить реализацию этой функции системе \zigzag. Сигнатура функции
представлена в листинге \ref{attrgetfunc}

\begin{lstlisting}[caption=Функция \myfunc{attr\_get()} - запрос атрибута. , label=attrgetfunc]
    #include    <zigzag.h>
    result_t attr_get(
        uint8_t     anum,
        void        *to
        );
\end{lstlisting}

{\bfseries Аргументы функции \myfunc{attr\_get()}}:

{\itshape
\begin{enumerate}
\item \myarg{anum} - номер запрашиваемого атрибута.
\item \myarg{to} - указатель на область памяти, в которую прикладной объект должен записать значение атрибута.
\end{enumerate}
}

{\bfseries Возвращаемое значение:}

{\itshape
Функция \myfunc{attr\_get()} должна возвратить одно из следующих значений:
\begin{itemize}
\item ENOERR - значение атрибута успешно записано по указателю \myarg{to}.
\item EINVAL - некорректный номер атрибута.
\item EACCESS - нарушен режим доступа к атрибуту.
\item ENOSYS - в текущей реализации атрибут не доступен.
\end{itemize}
}

\subsection{Установка атрибута}

Установка значения атрибута прикладного объекта осуществляется посредством вызова функции \myfunc{attr\_set()}. 
Прикладной объект обязан предоставить реализацию этой функции системе \zigzag. Сигнатура функции представлена
в листинге \ref{attrsetfunc}.

\begin{lstlisting}[caption=Функция \myfunc{attr\_set()} - установка атрибута. , label=attrsetfunc ]
    #include    <zigzag.h>
    result_t attr_set(
        uint8_t     anum,
        void        *from
        );
\end{lstlisting}

{\bfseries Аргументы функции \myfunc{attr\_set()}:}

{\itshape
\begin{enumerate}
\item \myarg{anum} - номер устанавливаемого атрибута.
\item \myarg{from} - указатель на область памяти, содержащей новое значение атрибута.
\end{enumerate}
}

{\bfseries Возвращаемое значение:}

{\itshape
Функция \myfunc{attr\_set()} должна возвратить одно из следующих значений:
\begin{itemize}
\item ENOERR - успешно установлено новое значение атрибута.
\item EINVAL - некорректный номер атрибута.
\item EACCESS - нарушен режим доступа к атрибуту. Запрещено изменение атрибута.
\item ENOSYS - в текущей реализации не реализована установка атрибута.
\end{itemize}
}

\section{Интерфейс обмена сообщениями}

\subsection{Создание нового сообщения}

Обмен данными между прикладными объектами осуществляется посредством передачи сообщений.
В листинге \ref{newmsgfunc} представлена сигнатура функции формирования нового сообщения.

\begin{lstlisting}[caption=Функция \myfunc{msg\_new()} - создание сообщения., label=newmsgfunc, numbers=left, frame=leftline]
    #include    <zigzag.h>
    msg_t   msg_new(
        net_addr_t  dst_addr,
        app_port_t  dst_port,
        uint8_t     msg_type,
        size_t      body_size
        );
\end{lstlisting}

{\bfseries Аргументы функции \myfunc{msg\_new()}:}

{\itshape
\begin{enumerate}
\item \myarg{dst\_addr} - сетевой адрес назначения,
\item \myarg{dst\_port} - прикладной порт назначения,
\item \myarg{msg\_type} - тип сообщения. Допустимые типы сообщений от 32 до 255.
\item \myarg{body\_size} - размер тела сообщения в октетах.
\end{enumerate}
}

{\bfseries Возвращаемое значение:}

{\itshape
Функция возращает идентификатор сообщения в случае успешного вызова, а именно значение $\geq 0$, либо
отрицательное значение в случае ошибки. Допустимы следующие ошибочные значения:
\begin{itemize}
\item EINVAL - некорректное значение аргумента,
\item ENOMEM - недостаточно памяти для создания нового сообщения.
\item ENOSYS - в текущей реализации функция не поддерживается.
\end{itemize}
}

После создания сообщение находится в заблокированном состоянии, то есть система \zigzag не
может производить над этим сообщением какие-либо дейтсвия ( например, отправку ). 

\subsection{Получение информации о сообщении}

Пока сообщение находится в заблокированном состоянии информация о нём может быть получена посредством
вызова функции \myfunc{msg\_info()}. Сигнатура этой функции представлена в листинге \ref{msginfofunc}.

\begin{lstlisting}[caption=Функция \myfunc{msg\_info()} - информация о сообщении, label=msginfofunc ]
    #include    <zigzag.h>
    result_t msg_info(
        msg_t   msg,  
        struct msginfo  *info
        );
\end{lstlisting}

{\bfseries Аргументы функции \myfunc{msg\_info()}:}

\begin{enumerate}
{\itshape
\item \myarg{msg} - идентификатор сообщения.
\item \myarg{info} - указатель на структуру, в которую будет скопирована информация о сообщении. Структура
определена в заголовочном файле zigzag.h. Определение структуры представлено в листинге \ref{msginfostruc}.
}

\begin{lstlisting}[caption=Определение структуры \myarg{msginfo}., label=msginfostruc ]
    struct msginfo {
        net_addr_t  dst_addr;
        app_port_t  dst_port;
        net_addr_t  src_addr;
        app_port_t  src_port;
        uint8_t     msg_type;
        size_t      body_size;
        void        *body_ptr;
    };
\end{lstlisting}

{\bfseries Описание полей структуры \myarg{msginfo}:}

{\itshape
\begin{itemize}
\item \myarg{dst\_addr} - сетевой адрес назначения,
\item \myarg{dst\_port} - прикладной порт назначения,
\item \myarg{src\_addr} - сетевой адрес источника,
\item \myarg{src\_port} - прикладной порт источника,
\item \myarg{msg\_type} - тип сообщения,
\item \myarg{body\_size} - размер тела сообщения в октетах,
\item \myarg{body\_ptr} - указатель на тело сообщения.
\end{itemize}
}

\end{enumerate}


{\bfseries Возвращаемые функцией \myfunc{msg\_info()} значения:}

{\itshape
\begin{itemize}
\item ENOERR - информация о сообщении успешно получена,
\item EINVAL - неправильный идентификатор сообщения или аргумент \myarg{info} равен нулю.
\item ENOSYS - в текущей реализации вызов функции не поддерживается.
\end{itemize}
}

\subsection{Удаление сообщения}

Удаление сообщения из системы осуществляется посредством функции \myfunc{msg\_destroy}. 
Сигнатура функции представлена в листинге \ref{msgdestroyfunc}.

\begin{lstlisting}[caption=Функция \myfunc{msg\_destroy()} - удаление сообщения., label=msgdestroyfunc ]
    #include    <zigzag.h>
    result_t msg_destroy(
        msg_t   msg
        );
\end{lstlisting}

{\bfseries Аргументы функции \myfunc{msg\_destroy()}:}

{\itshape
\begin{enumerate}
\item msg - идентификатор удаляемого сообщения.
\end{enumerate}
}

{\bfseries Возвращаемые функцией \myfunc{msg\_destroy()} значения:}

{\itshape
\begin{itemize}
\item ENOERR - сообщение успешно удалено из системы,
\item EINVAL - некорретный идентификатор сообщения,
\item EBUSY - системы производит над сообщением какие-то операции. Удалено может быть только 
заблокированное сообщение.
\item ENOSYS - в текущей версии системы функция не поддерживается.
\end{itemize}
}

\subsection{Передача сообщений}

Для передачи сообщения предназначена функция \myfunc{msg\_send()}. После вызова этой функции
сообщение разблокируется и система \zigzag начинает процедуру отправки этого сообщения по назначению.
Сигнатура функции \myfunc{msg\_send()} представлена в листинге \ref{msgsendfunc}.

\begin{lstlisting}[caption=Функция \myfunc{msg\_send()} - отправка сообщения., label=msgsendfunc ]
    #include    <zigzag.h>
    result_t msg_send(
        msg_t   msg
        );
\end{lstlisting}

{\bfseries Аргументы функции \myfunc{msg\_send()}:}

{\itshape
\begin{enumerate}
\item msg - идентификатор отправляемого сообщения.
\end{enumerate}
}

{\bfseries Возвращаемые функцией \myfunc{msg\_send()} значения:}

{\itshape
\begin{itemize}
\item ENOERR - успешно начата процедура отправки сообщения,
\item EINVAL - некорректное значение аргумента \myarg{msg},
\item ENOSYS - функция не поддерживается системой \zigzag.
\end{itemize}
}

После окончания процедуры отправки система \zigzag вызывает функцию \myfunc{msg\_send\_done()},
в свою очередь прикладной объект обязан предоставить реализацию этой функции. В листинге \ref{msgsenddonefunc} представлена 
сигнатура функции \myfunc{msg\_send\_done()}.

\begin{lstlisting}[caption=Функция \myfunc{msg\_send\_done()} - окончание отправки, label=msgsenddonefunc ]
    #include    <zigzag.h>
    void msg_send_done(
        msg_t   msg
        send_status_t   status
        );
\end{lstlisting}

{\bfseries Аргументы функции \myfunc{msg\_send\_done()}:}

{\itshape
\begin{enumerate}
\item \myarg{msg} - идентификатор сообщения,
\item \myarg{status} - статус завершения процедуры отправки сообщения. Допустимы следующие значения:
\begin{itemize}
\item STATUS\_SUCCESS - сообщение успешно отправлено по назначению,
\item STATUS\_TIMEOUT - продолжительность процедуры отправки превысила допустимый предел. Сообщение
не было отправлено,
\item STATUS\_MAX\_ATTEMPTS - превышено число попыток отправки сообщения. Сообщение не было отправлено по 
назначению.
\end{itemize}
\end{enumerate}
}
\subsection{Приём сообщений}

Приняв сообщение предназначенное данному прикладному объекту система \zigzag вызывает функцию
\myfunc{msg\_recv()}. В свою очередь прикладной объект обязан предоставить реализацию этой функции.
Сигнатура функции представлена в листинге \ref{msgrecvfunc}.

\begin{lstlisting}[caption=Функция \myfunc{msg\_recv()} - приём сообщения., label=msgrecvfunc ]
    #include    <zigzag.h>
    size_t msg_recv(
        msg_t   msg
        ) {}
\end{lstlisting}

{\bfseries Аргументы функции \myfunc{msg\_recv()}:}

{\itshape
\begin{enumerate} 
\item \myarg{msg} - идентификатор принятого сообщения.
\end{enumerate}
}

{\bfseries Возвращаемое значение:}

{\itshape
Функция обязана возвратить размер всего сообщения ( включая заголовок сообщения ) или 0, если размер
сообщения определить не удалось.
}

Перед вызовом функции \myfunc{msg\_recv()} сообщение переводится в заблокированное состояние. Поэтому прикладной
объект может воспользоваться функцией \myfunc{msg\_info()} для получения информации о сообщении.

Поскольку размер тела сообщения системе не известен, то поле \myarg{body \_size} структуры \myarg{msginfo} равно 0. При этом поле
\myarg{body\_ptr} указывает на первый октет тела сообщения. По этой же причине прикладной объект обязан сообщить
системе размер сообщения, возвратив его в функции \myfunc{msg\_recv()}. Если \myfunc{msg\_recv()} возвратит 
некорректное значение, то оставшиеся сообщения в пакете будут обработаны неправильно.

\section{Интерфейс доступа к хранилищу}
Хранилище предназначено для сохранения данных о каком-либо событии. Хранилище может использоваться для
обмена данными как внутри прикладного объекта, так и между различными объектами, которые при этом могут находиться
на разных узлах сети. 

\subsection{Запрос места в хранилище}

Для запроса места в хранилище предназначена функция \myfunc{storage\_alloc()}. Сигнатура этой функции 
представлена в листинге \ref{storageallocfunc}.

\begin{lstlisting}[caption=Функция \myfunc{storage\_alloc()} - запрос места в хранилище, label=storageallocfunc]
    #include    <zigzag.h>
    int32_t storage_alloc(
        size_t  size
        );
\end{lstlisting}

{\bfseries Аргументы функции \myfunc{storage\_alloc()}:}

{\itshape
\begin{enumerate} 
\item \myarg{size} - размер запрашиваемой памяти в хранилище.
\end{enumerate}
}

{\bfseries Возвращаемое значение:}

{\itshape
\begin{itemize}
\item Функция возвращает отрицательное значение в случае ошибки. При этом допустимы следующие значения:
    \begin{itemize}
        \item ENOMEM - недостаточно места в хранилище,
        \item EINVAL - некорректное значение аргумента \myarg{size},
        \item ENOSYS - функция не поддерживается в текущей версии системы.
    \end{itemize}
\item Если место в хранилище успешно выделено, возвращается неотрицательный дескриптор. Этот дескриптор впоследствии
может быть передан другим прикладным объектам в сети.
\end{itemize}
}

\subsection{Получение доступа к данным}

К одной выделенной области памяти в хранилище могут получить доступ несколько прикладных объектов. При этом
в любой момент времени либо несколько объектов могут иметь доступ на чтение, либо только один объект может иметь доступ на запись.

В листинге \ref{storagelockfunc} представлена функция получения доступа.

\begin{lstlisting}[caption=Функция \myfunc{storage\_lock()} - доступ к хранилищу, label=storagelockfunc ]
    #include    <zigzag.h>
    void *  storage_lock(
        int32_t storage_desc,
        uint8_t  mode
        );
\end{lstlisting}

{\bfseries Аргументы функции \myfunc{storage\_lock()}:}

{\itshape
\begin{enumerate} 
\item \myarg{storage\_desc} - дескриптор области памяти в хранилище.
\item \myarg{mode} - запрашиваемый режим доступа. Допустимы следующие значения:
    \begin{itemize}
        \item A\_READ - доступ только на чтение,
        \item A\_RW - доступ на чтение и запись.
    \end{itemize}
\end{enumerate}
}

{\bfseries Возвращаемое значение:}

{\itshape
Функция возвращает указатель на область памяти в случае успешного вызова или 0 в противном случае.
}

После завершения доступа к памяти хранилища должна быть вызвана функция \myfunc{storage\_unlock()}.
Сигнатура функции представлена в листинге \ref{storageunlockfunc}.

\begin{lstlisting}[caption=Функция \myfunc{storage\_unlock()} - завершение доступа, label=storageunlockfunc ]
    #include    <zigzag.h>
    result_t  storage_unlock(
        int32_t storage_desc
        );
\end{lstlisting}

{\bfseries Аргументы функции \myfunc{storage\_unlock()}:}

{\itshape
\begin{enumerate} 
\item \myarg{storage\_desc} - дескриптор области памяти в хранилище.
\end{enumerate}
}

{\bfseries Возвращаемые значения:}

{\itshape
\begin{itemize}
\item ENOERR - успешно выполнено завершение доступа к хранилищу,
\item EINVAL - некорректное значение аргумента \myarg{storage\_desc},
\item ENOSYS - в текущей реализации функция не поддерживается.
\end{itemize}
}

Заметим, что системы \zigzag извещает прикладные объекты узла о различных событиях, связанных с 
хранилищем, посредством интерфейса системных событий (см. раздел \ref{SysEventSect}).

\subsection{ Высвобождение места в хранилище }

Ранее выделенная область памяти может быть освобождена и возвращена в хранилище с помощью
вызова функции \myfunc{storage\_free()} ( см. листинг. \ref{storagefreefunc} ).

\begin{lstlisting}[caption=Функция \myfunc{storage\_free()} - возврат памяти в хранилище, label=storagefreefunc ]
    #include    <zigzag.h>
    result_t  storage_free(
        int32_t storage_desc
        );
\end{lstlisting}

{\bfseries Аргументы функции \myfunc{storage\_free()}:}

{\itshape
\begin{enumerate} 
\item \myarg{storage\_desc} - дескриптор области памяти в хранилище.
\end{enumerate}
}

{\bfseries Возвращаемые значения:}

{\itshape
\begin{itemize}
\item ENOERR - успешно выполнено освобождение области памяти в хранилище,
\item EINVAL - некорректное значение аргумента \myarg{storage\_desc},
\item ENOSYS - в текущей реализации функция не поддерживается.
\end{itemize}
}

\section{Интерфейс системных событий}
\label{SysEventSect}

С помощью данного интерфейса система \zigzag оповещает прикладной объект о различных событиях, происходящих в системе.
Прикладному объекту также предоставляется возможность сгенерировать определённые события. Эта возможность должна использоваться
прикладными объектами для завершения обработки прерывания. Самые необходимые действия по обработке прерывания объект может
проводить в самой функции обработки прерывания. После этого объект должен сгенерировать системное событие и продолжить выполнение
в обработчике события.

\subsection{Обработка событий}

Прикладной объект должен предоставить системе \zigzag реализацию функции \myfunc{event\_handler()}, сигнатура
которой представлена в листинге \ref{eventhandlerfunc}.

\begin{lstlisting}[caption=Функция \myfunc{event\_handler()} - обработчик событий., label=eventhandlerfunc ]
    #include    <zigzag.h>
    void  event_handler(
        event_type_t    event_type,
        unidata_t       unidata
        );
\end{lstlisting}

{\bfseries Аргументы функции \myfunc{event\_handler()}:}
\begin{enumerate}

{\itshape
\item \myarg{event\_type} - тип события. Системой \zigzag зарезервированы типы событий из диапазона от 0xe0 до 0xff. На 
данный момент используются события представленные в таблице \ref{eventtypestab}.
}

\tablecaption{Системные типы событий.}
\tablefirsthead{\hline Тип события & Номер события & Описание \\ \hline}
%\tablehead{\hline Тип события & Номер события & Описание \\ \hline}
\tabletail{\hline}
\tablelasttail{\hline}
\begin{supertabular}{|l|p{2cm}|p{8cm}|}
\label{eventtypestab}

   EV\_JOIN & 0xff & Узел присоединился к сети \\\hline
   EV\_LEAVE & 0xfe & Узел покинул сеть \\\hline
   EV\_SLEEP & 0xfd & Узел переходит в состояние сна. Аргумент \myarg{unidata} содержит предполагаемую продолжительность сна в миллисекундах. \\\hline
   EV\_SFREE & 0xfc & Освобождена область памяти хранилища. Аргумент \myarg{unidata} содержит дескриптор этой области памяти. \\\hline
   EV\_RUNLOCK & 0xfb & Завершён доступ на чтение к области памяти хранилища. Аргумент \myarg{unidata} равен дескриптору области памяти. \\\hline
   EV\_WUNLOCK & 0xfa & Завершён доступ на запись к области памяти хранилища. Аргумент \myarg{unidata} равен дескриптору области памяти. \\\hline
\end{supertabular}

{\itshape
\item \myarg{unidata} - аргумент содержит дополнительные данные о событии.
}

\end{enumerate}

\subsection{Генерация событий }

При помощи вызова функции \myfunc{event\_emit()} прикладной объект просит систему \zigzag сгенерировать событие и вызвать обработчик
событий \myfunc{event\_handler()}. При возникновении аппаратного прерывания и после его предварительной обработки объект должен вызвать
функцию \myfunc{event\_emit()} для продолжения обработки прерывания уже в функции \myfunc{event\_handler()}. Сигнатура функции \myfunc{event\_emit()}
представлена в листинге \ref{eventemitfunc}.

\begin{lstlisting}[caption=Функция \myfunc{event\_emit()} - генерация события., label=eventemitfunc ]
    #include    <zigzag.h>
    result_t  event_emit(
        priority_t      priority,
        event_type_t    event_type,
        unidata_t       unidata
        );
\end{lstlisting}

{\bfseries Аргументы функции \myfunc{event\_emit()}:}

{\itshape
\begin{enumerate}
\item \myarg{priority} - приоритет сгенерированного события перед всеми остальными событиями. Чем больше значение аргумента, тем выше
приоритет. Допустимые значения аргумента из диапазона от 0 до 127.
\item \myarg{event\_type} - тип события. Разрешено использовать типы событий из диапазона от 0x00 до 0xdf.
\item \myarg{unidata} - значение, которое будет передано в соответствующий вызов функции \myfunc{event\_handler}.
\end{enumerate}
}

\section{Интерфейс аппаратных прерываний}
\subsection{Обработка прерываний}

Обработка аппаратных прерываний осуществляется в функции \myfunc{irq\_ handler()}. Прикладной объект обязан 
предоставить реализацию этой функции. Сигнатура функции представлена в листинге \ref{irqhandlerfunc}.

\begin{lstlisting}[caption=Функция \myfunc{irq\_handler()} - обработчик прерываний., label=irqhandlerfunc ]
    #include    <zigzag.h>
    void  irq_handler(
        irq_t   irq
        ) {}
\end{lstlisting}

{\bfseries Аргументы функции \myfunc{irq\_handler()}:}

{\itshape
\begin{enumerate}
\item \myarg{irq} - номер прерывания. Зависит от используемого микроконтроллера. Так для TI msp430 определены
следующие номера прерываний ( в скобках указан адрес вектора прерывания ):
    \begin{itemize}
        \item 0 - DAC/DMA ( 0xffe0 ),
        \item 2 - PORT2 ( 0xffe2 ),
        \item 8 - PORT1 ( 0xffe8 ),
        \item 14 - ADC 12 ( 0xffee )
        \item 22 - COMPARATOR A ( 0xfff6 )
    \end{itemize}
\end{enumerate}
}

Из функции \myfunc{irq\_handler} разрешено вызывать следующие функции системы \zigzag:
\begin{itemize}
\item \myfunc{event\_emit()};
\item \myfunc{storage\_alloc()}, \myfunc{storage\_destroy()};
\item \myfunc{storage\_lock()},  \myfunc{storage\_unlock()};
\item \myfunc{sys\_time()};
\item \myfunc{timer\_set()}, \myfunc{timer\_info()}, \myfunc{timer\_stop()}.
\end{itemize}

\section{Интерфейс системного таймера}

Данный интерфейс позволяет воспользоваться возможностями существующего в системе аппаратного таймера. Число
доступных таймеров зависит от аппаратной конфигурации узла.

\subsection{Значение счётчика таймера}

Функция \myfunc{timer\_counter()} возвращает текущее значение счётчика таймера. Частота обновления счётчика
зависит от аппаратной конфигурации узла. В листинге \ref{timercounterfunc} представлена сигнатура функции
\myfunc{timer\_ counter()}.

\begin{lstlisting}[caption=Функция \myfunc{timer\_counter()} - значение счётчика., label=timercounterfunc ]
    #include    <zigzag.h>
    uint32_t    timer_counter()
\end{lstlisting}

\subsection{Установка таймера}

Установить таймер на определённое время срабатывания позволяет функция \myfunc{timer\_set()}. Сигнатура
этой функции представлена в листинге \ref{timersetfunc}.

\begin{lstlisting}[caption=Функция \myfunc{timer\_set()} - установка таймера., label=timersetfunc ]
    #include    <zigzag.h>
    result_t    timer_set(
        uint8_t     tnum,
        uint32_t    tpoint
        );
\end{lstlisting}

{\bfseries Аргументы функции \myfunc{timer\_set()}:}

{\itshape
\begin{enumerate}
\item \myarg{tnum} - номер аппаратного таймера. Число доступных таймеров зависит от реализации.
\item \myarg{tpoint} - момент времени срабатывания таймера. Когда \myarg{tpoint} будет равен счётчику
таймера произойдёт срабатывание. При повторном вызове \myfunc{timer\_set()} таймер устанавливается
на новое значение \myarg{tpoint}.
\end{enumerate}
}

{\bfseries Возвращаемые значения:}

{\itshape 
\begin{itemize}
\item ENOERR - таймер успешно установлен,
\item EINVAL - некорректное значение аргумента. Возможно неправильно указан номер таймера.
\item ENOSYS - функция не поддреживается системой.
\end{itemize}
}

\subsection{Срабатывание таймера}

Если таймер был установлен посредством функции \myfunc{timer\_set()} и \myarg{tpoint}$\geq$\myfunc{timer\_count()}
происходит срабатывание таймера и система вызывает функцию \myfunc{timer\_fired()}. Прикладной объект обязан
предоставить реализацию этой функции. Сигнатура функции \myfunc{timer\_fired()} представлена в листинге \ref{timerfiredfunc}.

\begin{lstlisting}[caption=Функция \myfunc{timer\_fired()} - срабатывание таймера., label=timerfiredfunc ]
    #include    <zigzag.h>
    void    timer_fired(
        uint8_t     tnum
        );
\end{lstlisting}

{\bfseries Аргументы функции \myfunc{timer\_fired()}:}

{\itshape
\begin{enumerate}
\item \myarg{tnum} - номер сработавшего аппаратного таймера.
\end{enumerate}
}

Из функции \myfunc{timer\_fired()} разрешено вызывать те же функции, что и из \myfunc{irq\_handler()}.

\subsection{Получение информации о таймере}

Узнать установлен ли таймер и на какое значение  можно с помощью функции \myfunc{timer\_info()}, сигнатура
которой представлена в листинге \ref{timerinfofunc}.

\begin{lstlisting}[caption=\myfunc{timer\_info()} - информация о таймере, label=timerinfofunc ]
    #include    <zigzag.h>

    result_t    timer_info(
        uint8_t     tnum,
        struct timerinfo    *tinfo
        );
\end{lstlisting}

{\bfseries Аргументы функции \myfunc{timer\_info()}:}

\begin{enumerate}
{\itshape
\item \myarg{tnum} - номер таймера, информацию о котором требуется получить.
\item \myarg{tinfo} - указатель на структуру, в которую записывается информация о таймере. Структура определена в заголовочном
файле zigzag.h. Определение структуры представлено в листинге \ref{timerinfostruc}.
}

\begin{lstlisting}[caption=Определение структуры \myarg{timerinfo}, label=timerinfostruc]
    struct timerinfo {
        uint8_t     is_set;
        uint32_t    tpoint;
    };
\end{lstlisting}

{\bfseries Описание полей структуры:}

{\itshape
\begin{enumerate}
\item \myarg{is\_set} Если поле равно 0, то таймер не установлен, иначе если \myarg{is\_set} > 0, то таймер установлен.
\item \myarg{tpoint} - значение счётчика, при достижении которого должно произойти срабатывание таймера в случае если он установлен.
\end{enumerate}
}
\end{enumerate}

\subsection{Остановка таймера}

    Остановить установленный таймер позволяет функция \myfunc{timer\_stop()}.
В листинге \ref{timerstopfunc} представлена сигнатура этой функции.

\begin{lstlisting}[caption=\myfunc{timer\_stop()} - остановка таймера, label=timerstopfunc ]
    #include    <zigzag.h>

    result_t    timer_stop(
        uint8_t     tnum
        );
\end{lstlisting}

{\bfseries Аргументы функции \myfunc{timer\_stop()}:}

{\itshape
\begin{enumerate}
\item \myarg{tnum} - номер таймера, который требуется остановить.
\end{enumerate}
}

{\bfseries Возвращаемые значения:}

{\itshape
\begin{itemize}
\item ENOERR - таймер успешно остановлен,
\item EINVAL - некорректное значение аргумента \myarg{tnum},
\item ENOSYS - в текущей реализации функция не поддерживается.
\end{itemize}
}

\section{Интерфейс общесистемных функций}

\subsection{Инициализация объекта}

Функция \myfunc{sys\_init()} вызывается системой \zigzag при запуске и предназначена
для инициализации внутренних структур данных прикладного объекта. Сигнатура функции
представлена в листинге \ref{sysinitfunc}.

\begin{lstlisting}[caption=\myfunc{sys\_init()} - инициализация системы, label=sysinitfunc ]
    void    sys_init() {}
\end{lstlisting}

Прикладной объект обязан предоставить системе \zigzag реализацию этой функции.

\subsection{Запрос текущего времени}

Функция \myfunc{sys\_time()} возвращает локальное время узла в миллисекундах с 1 января 1970 года.
Все временные метки, отправляемые в сообщениях, должны быть получены с помощью этой функции.
Сигнатура функции представлена в листинге \ref{systimefunc}.

\begin{lstlisting}[caption=\myfunc{sys\_time()} - текущее время, label=systimefunc ]
    uint64_t    sys_time();
\end{lstlisting}


